\documentclass[a4paper,12pt]{article}
\usepackage[T1]{fontenc}
\begin{document}
\section{Kompilacja}
Kompilacja dla każdego zadania wygląda tak samo - w folderze src należy wykonać \verb|javac *.java|,
a następnie rozpocząć program poleceniem \verb|java Main|.
\section{Zadania}
\paragraph{Zadanie 1}
Zaimplementowałem kolejkę oraz stos. Po włączeniu programu tworzy on stos o rozmiarze 50 i próbuje dodać do niego 53 elementy, a potem je wyjąć. 
To samo robi z kolejką.

\paragraph{Zadanie 2, 3}
Zaimplementowałem listę cykliczną jednokierunkową oraz dwukierunkową. Oba programy robią to samo -- tworzą dwie 10-elementowe listy, wypisują je, 
po czym łączą je, i znów wypisują obie. Na końcu wypisują średnią liczbę porównań podczas testu opisanego w zadaniu. Widzimy, że w przypadku obu rodzajów list te wartośći są bardzo podobne.
\end{document}