\documentclass{article}

\usepackage{graphicx}
\usepackage[T1]{fontenc}
\usepackage[a4paper, total={6in, 9in}]{geometry}
\usepackage{float}

\graphicspath{{img/}}
\title{Lista 1}
\author{Krzysztof Kleszcz}
\date{29.10.2024}

\begin{document}

\maketitle

\section{Wprowadzenie}
Wszystkie przedstawione poniżej całki zostały oszacowane metodą probabilistyczną podaną w zadaniu. 
Do przeprowadzenia symulacji został użyty generator liczb losowych \emph{Mersenne Twister}.
\paragraph{Implementacja.}
Algorytm został zaimplementowany w języku programowania \emph{Java}.
Plik \verb|L1.jar| jest skompilowany i gotowy do uruchomienia poleceniem 
\begin{center}
    \begin{verbatim}

        java -jar L1.jar a b > out.txt
    \end{verbatim}
\end{center}
Parametry \verb|a| i \verb|b| to odpowiednio wybór szacowanej całki \begin{math}
    (0 - 3)
\end{math} 
oraz liczba powtórzeń algorytmu \emph{k}.
Szczegóły użycia parametrów znajdują się w komentarzach kodu źródłowego, w pliku \verb|App.java|.
Plik \verb|out.txt| to wybrane przeze mnie wyjście programu.

\paragraph{Wizualizacja.}
Dane wygenerowane przez program w \emph{Javie} zostały wprowadzone do skryptu w języku \emph{Python} oraz za pomocą 
biblioteki \emph{matplotlib} ukazane na wykresach. Skrypt znajduje się w pliku \verb|main.py|.
\newline

Punkty o kolorze niebieskim oznaczają wyniki pojedynczych testów, a punkty o kolorze czerwonym to średnia wartość
wyników wszystkich testów dla danej liczby losowo wygenerowanych punktów \emph{n}. 
Zielona prosta to rzeczywista wartość szacowanej całki.
\pagebreak

\section{Przykłady}

\paragraph{A}
Szacowanie całki oznaczonej

\[    \int_{0}^{8}  \sqrt[3]{x} \,dx \]
\begin{figure}[H]
    \centering
    \includegraphics[width=0.7\textwidth]{1x5.png}
    \begin{center}
        (a) k = 5
    \end{center}
\end{figure}

\begin{figure}[H]
    \centering
    \includegraphics[width=0.7\textwidth]{1x50.png}
    \begin{center}
        (b) k = 50
    \end{center}
\end{figure}

\pagebreak

\paragraph{B}
Przybliżenie całki oznaczonej

\[    \int_{0}^{\pi }  \sin (x) \,dx \]

\begin{figure}[H]
    \centering
    \includegraphics[width=0.75\textwidth]{2x5.png}
    \begin{center}
        (a) k = 5
    \end{center}
\end{figure}

\begin{figure}[H]
    \centering
    \includegraphics[width=0.75\textwidth]{2x50.png}
    \begin{center}
        (b) k = 50
    \end{center}
\end{figure}

\pagebreak

\paragraph{C}
Przybliżenie całki oznaczonej

\[    \int_{0}^{1}  4x(1-x)^3 \,dx \]

\begin{figure}[H]
    \centering
    \includegraphics[width=0.75\textwidth]{3x5.png}
    \begin{center}
        (a) k = 5
    \end{center}
\end{figure}

\begin{figure}[H]
    \centering
    \includegraphics[width=0.75\textwidth]{3x50.png}
    \begin{center}
        (b) k = 50
    \end{center}
\end{figure}

\pagebreak

\paragraph{D}
Przybliżenie \begin{math}
    \pi
\end{math}
za pomocą całki oznaczonej 

\[    \int_{-1}^{1}  2\sqrt{1-x^2} \,dx \]

\begin{figure}[H]
    \centering
    \includegraphics[width=0.75\textwidth]{PIx5.png}
    \begin{center}
        (a) k = 5
    \end{center}
\end{figure}

\begin{figure}[H]
    \centering
    \includegraphics[width=0.75\textwidth]{PIx50.png}
    \begin{center}
        (b) k = 50
    \end{center}
\end{figure}

\pagebreak

\section{Podsumowanie}
Ten probablisityczny sposób szacowania całek oznaczonych jest dość skuteczny. Dla liczby testów \begin{math}
    k = 5
\end{math}
średnie wyniki dla najwyższych wartości \begin{math}
    n
\end{math}
znajdują się w granicach \begin{math}
    \pm 1\%
\end{math}
rzeczywistej wartości całki. Dla \begin{math}
    k = 50
\end{math}
już od \begin{math}
    n \approx 500
\end{math}
otrzymujemy precyzję bliską \begin{math}
    \pm 1\%
\end{math},
a dla najwyższych \begin{math}
    n
\end{math}
precyzja osiąga około \begin{math}
    \pm 0.1\%
\end{math}.



\end{document}