\documentclass{article}

\usepackage{graphicx}
\usepackage[T1]{fontenc}
\usepackage[a4paper, total={6in, 9in}]{geometry}
\usepackage{float}

\title{Lista 4}
\author{Krzysztof Kleszcz}
\date{27.01.2025}

\begin{document}

\graphicspath{{zadania/img}}

\maketitle

\section{Wprowadzenie}
Wszystkie symulacje i obliczenia zostały przeprowadzone w \emph{Pythonie}. Pliki znajdują się w folderze \verb|zadania|.

\section{Zadania}

\paragraph{Zadanie 1}
Szacowanie "ogonów" rozkładu dwumianowego za pomocą nierówności Markowa i Czebyszewa.
\begin{figure}[H]
    \centering
    \includegraphics[width=0.7\textwidth]{zad1.png}
\end{figure}
Jak widać na powyższej tabeli, oszacowane wyniki z nierówności Markowa i Czebyszewa są dosyć daleko od dokładnych wartości. Widać jednak, że nierówność Czebyszewa staje się dokładniejsza wraz z wzrostem parametru $N$.

\paragraph{Zadanie 2}
Błądzenie losowe na liczbach całkowitych.
\begin{figure}[H]
    \centering
    \includegraphics[width=0.7\textwidth]{zad2_5.png}
\end{figure}
\begin{figure}[H]
    \centering
    \includegraphics[width=0.7\textwidth]{zad2_10.png}
\end{figure}
\begin{figure}[H]
    \centering
    \includegraphics[width=0.7\textwidth]{zad2_15.png}
\end{figure}
\begin{figure}[H]
    \centering
    \includegraphics[width=0.7\textwidth]{zad2_20.png}
\end{figure}
\begin{figure}[H]
    \centering
    \includegraphics[width=0.7\textwidth]{zad2_25.png}
\end{figure}
\begin{figure}[H]
    \centering
    \includegraphics[width=0.7\textwidth]{zad2_30.png}
\end{figure}

Jak widać, wraz ze zwiększającym się $N$ rozkład normalny zbliża się do histogramu wyliczeń empirycznych. Dystrybuanta zdaje się być blisko do prawdziwej wartości dla każdego N.

\begin{figure}[H]
    \centering
    \includegraphics[width=0.7\textwidth]{zad2_100.png}
\end{figure}

Dla $N = 100$ rozkład normalny pokrywa się prawie idealnie z histogramem.

\paragraph{Zadanie 3}
Błądzenie losowe cz. 2 -- "czas spędzony nad osią OX"

\begin{figure}[H]
    \centering
    \includegraphics[width=0.7\textwidth]{zad3_100.png}
\end{figure}
\begin{figure}[H]
    \centering
    \includegraphics[width=0.7\textwidth]{zad3_1000.png}
\end{figure}
\begin{figure}[H]
    \centering
    \includegraphics[width=0.7\textwidth]{zad3_10000.png}
\end{figure}

Histogramy w tym zadaniu są skoncentrowane na brzegach wykresów. Na "chłopski rozum" ma to sens -- gdy im dalej linia jest "nad" osią X, tym jest mniejsza szansa że spadnie poniżej tę oś, i vice versa. Dla $N = 100$ kolumny histogramu są poszarpane i nieregularne, ale dla większych $N$ przypominają już ładne doliny.

\begin{figure}[H]
    \centering
    \includegraphics[width=0.7\textwidth]{zad3_arcsin.png}
\end{figure}

Szczyty histogramów blisko pokrywają się z wykresem rozkładu arcusa sinusa.

\end{document}