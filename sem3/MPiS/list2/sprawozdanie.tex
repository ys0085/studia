\documentclass{article}

\usepackage{graphicx}
\usepackage[T1]{fontenc}
\usepackage[a4paper, total={6in, 9in}]{geometry}
\usepackage{float}

\graphicspath{{img/}}
\title{Lista 2}
\author{Krzysztof Kleszcz}
\date{29.10.2024}

\begin{document}

\maketitle

\section{Wprowadzenie}
Wszystkie przedstawione poniżej całki zostały oszacowane metodą probabilistyczną podaną w zadaniu. 
Do przeprowadzenia symulacji został użyty generator liczb losowych \emph{Mersenne Twister}.
\paragraph{Implementacja.}
Algorytm został zaimplementowany w języku programowania \emph{Java}.
Plik \verb|Program.jar| jest skompilowany i gotowy do uruchomienia poleceniem 
\begin{center}
    \begin{verbatim}

        java -jar Program.jar > data.txt
    \end{verbatim}
\end{center}
Plik \verb|data.txt| to dane wygenerowane przez program.

\paragraph{Wizualizacja.}
Dane wygenerowane przez program w \emph{Javie} zostały wprowadzone do skryptu w języku \emph{Python} oraz za pomocą 
biblioteki \emph{matplotlib} ukazane na wykresach. Skrypt znajduje się w pliku \verb|main.py|.
\newline

Punkty o kolorze niebieskim oznaczają wyniki pojedynczych testów, a punkty o kolorze czerwonym to średnia wartość
wyników wszystkich testów dla danej liczby urn \emph{n}. 
\pagebreak

\section{Wyniki}

\paragraph{A}
Paradoks urodzinowy (ang. \emph{Birthday paradox}), czyli kula która pierwsza wpada do niepustej urny.

\begin{figure}[H]
    \centering
    \includegraphics[width=0.7\textwidth]{Bn.png}
    \begin{center}
        (a) Wykres \begin{math}
            B_n
        \end{math} od \begin{math}
            n
        \end{math}.
    \end{center}
\end{figure}

Jak widać na wykresie (a), wartość $B_n$ jest słabo skoncentrowana. Jej średnie wartości układają się w kształt krzywej,
która przypomina 
\begin{math}
    f(x) = \sqrt{x}
\end{math}.

\begin{figure}[H]
    \centering
    \includegraphics[width=0.7\textwidth]{avgB_over_n.png}
    \begin{center}
        (b) Wykres \begin{math}
            \frac{b(n)}{n}
        \end{math} od \begin{math}
            n
        \end{math}.
    \end{center}
\end{figure}

\begin{figure}[H]
    \centering
    \includegraphics[width=0.7\textwidth]{avgB_over_sqrt_n.png}
    \begin{center}
        (c) Wykres \begin{math}
            \frac{b(n)}{\sqrt{n}}
        \end{math} od \begin{math}
            n
        \end{math}.
    \end{center}
\end{figure}

Wykres (c) potwierdza to podejrzenie, widać, że wartości ilorazu \begin{math}
    \frac{b(n)}{\sqrt{n}}
\end{math}
trzymają się mniej więcej w zakresie $(1.1; 1.5)$. Można więc postawić hipotezę, że wartość \begin{math}
    b(n)
\end{math} rośnie asymptotycznie z $\sqrt{x}$.

\paragraph{}

Nazwa \emph{birthday paradox} pochodzi od innej wersji tego samego problemu -- zamiast wrzucać $m$ kul do $n$ urn, 
wybieramy osoby i przydzielamy im losowo urodziny -- stąd \emph{birthday} -- i sprawdzamy, 
kiedy dwie osoby będą miały urodziny tego samego dnia. Ta wersja problemu jest jednoznaczna z wrzucaniem 
m kul do 365 urn i obserwowaniu, czy dwie kule znajdują się w jednej urnie. 
Nie jest to jednak prawdziwy "paradoks".
Słowo \emph{paradox} bierze się z nieintuicyjnego wyników tego problemu -- okazuje się, że już dla 23 osób szansa, 
że istnieje para osób z tą samą datą urodzin, wynosi 50\%. Dla 70 osób ta szansa wynosi aż 99,9\%.


\paragraph{}

Paradoksu urodzinowego można używać również w kryptografii, np. w testowaniu funkcji haszujących. 
Tzw. atak urodzinowy (ang. \emph{birthday attack}) pomaga w wykrywaniu kolizji w funkcjach haszujących poprzez generowanie 
i haszowanie kolejnych wiadomości i sprawdzanie, czy dwa hasze są takie same.

\pagebreak

\paragraph{B}

Wartość $U_n$, czyli liczba pustych urn po wrzuceniu $n$ kul (tyle kul co urn).

\begin{figure}[H]
    \centering
    \includegraphics[width=0.7\textwidth]{Un.png}
    \begin{center}
        (a) Wykres \begin{math}
            U_n
        \end{math} od \begin{math}
            n
        \end{math}.
    \end{center}
\end{figure}

Jak widać na wykresie (a), wartość $U_n$ jest mocno skoncentrowana. Zdaje się rosnąć liniowo.

\begin{figure}[H]
    \centering
    \includegraphics[width=0.7\textwidth]{avgU_over_n.png}
    \begin{center}
        (b) Wykres \begin{math}
            \frac{u(n)}{n}
        \end{math} od \begin{math}
            n
        \end{math}.
    \end{center}
\end{figure}

Wykres (b) potwierdza tę intuicję -- wartość $\frac{u(n)}{n}$ jest mniej więcej stała, równa około $0.368$.
Ta liczba jest podejrzanie blisko do wartości $\frac{1}{e} = 0.36788...$ -- można postwić hipotezę, że 
\begin{math}
    u(n) = \frac{1}{e}  n.
\end{math}

\pagebreak

\paragraph{C}

Wartość $C_n$, czyli kula, po której wrzuceniu nie ma żadnej pustej urny -- tzw. problem kolekcjonera kuponów 
(ang. \emph{coupon collector's problem}).

\begin{figure}[H]
    \centering
    \includegraphics[width=0.7\textwidth]{Cn.png}
    \begin{center}
        (a) Wykres \begin{math}
            C_n
        \end{math} od \begin{math}
            n
        \end{math}.
    \end{center}
\end{figure}

Wartości na wykresie (a) nie są mocno rozproszone, lecz ich rozproszenie zwiększa się dla większych $n$.
Na pierwszy rzut oka $C_n$ zdaje się rosnąć liniowo.

\begin{figure}[H]
    \centering
    \includegraphics[width=0.7\textwidth]{avgC_over_n.png}
    \begin{center}
        (b) Wykres \begin{math}
            \frac{c(n)}{n}
        \end{math} od \begin{math}
            n
        \end{math}.
    \end{center}
\end{figure}

\begin{figure}[H]
    \centering
    \includegraphics[width=0.7\textwidth]{avgC_over_n_ln_n.png}
    \begin{center}
        (c) Wykres \begin{math}
            \frac{c(n)}{n \ln n}
        \end{math} od \begin{math}
            n
        \end{math}.
    \end{center}
\end{figure}

\begin{figure}[H]
    \centering
    \includegraphics[width=0.7\textwidth]{avgC_over_n2.png}
    \begin{center}
        (d) Wykres \begin{math}
            \frac{c(n)}{n^2}
        \end{math} od \begin{math}
            n
        \end{math}.
    \end{center}
\end{figure}

Na wykresach (b, c, d) widać, że wartość $\frac{c(n)}{n \ln n}$ jest mniej więcej stała. Można postawić hipotezę,
że $c(n)$ rośnie asymptyotycznie do $n \ln n$.

\paragraph{}
Nazwa "problem kolekcjonera kuponów" wynika z innej postaci tego problemu -- możemy sobie wyobrazić, 
że kolekcjoner szuka łącznie $n$ rozróżnialnych kuponów do swojej kolekcji. Każdy kupon, 
który dodaje do swojej kolekcji jest losowy (kolekcjoner może go już posiadać). 
$c(n)$ to liczba kuponów kolekcjoner dodał do swojej kolekcji w momencie, gdy znalazł wszystkie kupony.

\pagebreak

\paragraph{D}

Wartość $D_n$, czyli kula, po której wrzuceniu w każdej urnie są co najmniej dwie kule. -- tzw. brat kolekcjonera kuponów 
(ang. \emph{coupon collector's brother}).

\begin{figure}[H]
    \centering
    \includegraphics[width=0.7\textwidth]{Dn.png}
    \begin{center}
        (a) Wykres \begin{math}
            D_n
        \end{math} od \begin{math}
            n
        \end{math}.
    \end{center}
\end{figure}

Wykres (a) wygląda bardzo podobnie do wykresu $C_n$, jest podobnie rozproszony. Również wygląda, jak by rósł liniowo.

\begin{figure}[H]
    \centering
    \includegraphics[width=0.7\textwidth]{avgD_over_n.png}
    \begin{center}
        (b) Wykres \begin{math}
            \frac{d(n)}{n}
        \end{math} od \begin{math}
            n
        \end{math}.
    \end{center}
\end{figure}

\begin{figure}[H]
    \centering
    \includegraphics[width=0.7\textwidth]{avgD_over_n_ln_n.png}
    \begin{center}
        (c) Wykres \begin{math}
            \frac{d(n)}{n \ln n}
        \end{math} od \begin{math}
            n
        \end{math}.
    \end{center}
\end{figure}

\begin{figure}[H]
    \centering
    \includegraphics[width=0.7\textwidth]{avgD_over_n2.png}
    \begin{center}
        (d) Wykres \begin{math}
            \frac{d(n)}{n^2}
        \end{math} od \begin{math}
            n
        \end{math}.
    \end{center}
\end{figure}

Ciężko jest postawić hipotezę na temat wartości $d(n)$. Nie rośnie ona liniowo (b), ani kwadratowo (d).
Wykres (c) lekko maleje, co może być błędem pomiaru, lecz nie można z niego wyciągnąć dobrej hipotezy.
Można jedynie zgadywać, że $d(n)$ rośnie asymptotycznie do $n \ln n$ lub ewentualnie $n \ln \ln n$.

\pagebreak

\paragraph{E}

Wartość $D_n - C_n$, czyli ilość kul od momentu, gdy każda urna ma co najmniej jedną kulę do momentu, gdy każda urna ma co najmniej dwie kule.

\begin{figure}[H]
    \centering
    \includegraphics[width=0.7\textwidth]{Dn_minus_Cn.png}
    \begin{center}
        (a) Wykres \begin{math}
            D_n
        \end{math} od \begin{math}
            n
        \end{math}.
    \end{center}
\end{figure}

Wykres (a) jest mocno rozproszony. Ciężko określić jego tempo wzrostu, lecz wygląda na liniowy.

\begin{figure}[H]
    \centering
    \includegraphics[width=0.7\textwidth]{avgD_minus_C_over_n.png}
    \begin{center}
        (b) Wykres \begin{math}
            \frac{d(n) - c(n)}{n}
        \end{math} od \begin{math}
            n
        \end{math}.
    \end{center}
\end{figure}

\begin{figure}[H]
    \centering
    \includegraphics[width=0.7\textwidth]{avgD_minus_C_over_n_ln_n.png}
    \begin{center}
        (c) Wykres \begin{math}
            \frac{d(n) - c(n)}{n \ln n}
        \end{math} od \begin{math}
            n
        \end{math}.
    \end{center}
\end{figure}

\begin{figure}[H]
    \centering
    \includegraphics[width=0.7\textwidth]{avgD_minus_C_over_n_ln_ln_n.png}
    \begin{center}
        (d) Wykres \begin{math}
            \frac{d(n) - c(n)}{n \ln \ln n}
        \end{math} od \begin{math}
            n
        \end{math}.
    \end{center}
\end{figure}

Z danych z wykresów (b, c, d) jest trudno określić tempo wzrostu $d(n) - c(n)$. Wykres (c) zdaje się maleć, 
lecz wykresy (b, d) oba wyglądają na stałe. 
Można postawić następujące hipotezy -- $d(n) - c(n)$ rośnie asymptotycznie z $n$ lub z $n \ln \ln n$.


\end{document}

