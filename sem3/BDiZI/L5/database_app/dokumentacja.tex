\documentclass[a4paper,12pt]{article}
\usepackage{graphicx}
\usepackage[T1]{fontenc}
\usepackage[utf8]{inputenc}
\usepackage[polish]{babel}
\usepackage{geometry}
\usepackage{amsmath}
\usepackage{amssymb}
\usepackage{hyperref}
\geometry{margin=1in}

\title{Dokumentacja Bazodanowa}
\author{}
\date{}

\begin{document}

\maketitle

\tableofcontents
\newpage

\section{Opis celu aplikacji}

\subsection{Cel aplikacji}
Aplikacja umożliwia dodawanie produktów i klientów do bazy danych, oraz tworzenie zamówień oraz ich raportów na podstawie dostępnych produktów.

\subsection{Charakterystyka użytkowników}
\begin{itemize}
    \item Dodawanie i usuwanie produktów.
    \item Zarządzanie użytkownikami i zamówieniami.
    \item Generowanie raportów sprzedaży.
    \item Przeglądanie katalogu produktów.
    \item Składanie zamówień.
    \item Sprawdzanie historii zamówień.
    
\end{itemize}

\section{Modelowanie logiczne}

\subsection{Opis encji}
\begin{itemize}
    \item \textbf{Klienci:} \texttt{client\_id (PK)}, \texttt{first\_name}, \texttt{last\_name}, \texttt{email}.
    \item \textbf{Produkty:} \texttt{product\_id (PK)}, \texttt{name}, \texttt{description}, \texttt{price}, \texttt{stock}.
    \item \textbf{Zamówienia:} \texttt{order\_id (PK)}, \texttt{client\_id (FK)}, \texttt{Order\_date}, \texttt{status}.
    \item \textbf{Szczegóły zamówień:} \texttt{order\_detail\_id (PK)}, \texttt{order\_id (FK)}, \texttt{product\_id (FK)}, \texttt{count}, \texttt{price}.
    \item \textbf{Płatności:} \texttt{payment\_id (PK)}, \texttt{order\_id (FK)}, \texttt{amount}, \texttt{payment\_type}, \texttt{payment\_date}.
    \item \textbf{Raporty:} \texttt{report\_id (PK)}, \texttt{date}, \texttt{order\_count}, \texttt{total\_income}.
\end{itemize}

\pagebreak

\subsection{Diagram ER}
Ten diagram ER przedstawia relacje między tabelami.
\newline

\includegraphics[scale=0.4]{ER diagram.png}

\section{Normalizacja bazy danych}

\subsection{Proces normalizacji}
\begin{enumerate}
    \item \textbf{1NF:} Wszystkie tabele mają atrybuty atomowe.
    \item \textbf{2NF:} Wszystkie atrybuty są zależne od klucza głównego (brak redundancji w szczegółach zamówień i płatnościach).
    \item \textbf{3NF:} Brak zależności przechodnich (np. dane użytkowników są w osobnej tabeli, nie są trzymane w zamówieniach).
\end{enumerate}

\section{Klucze w relacjach}

\subsection{Tabela Zamówienia}
\begin{itemize}
    \item \textbf{Klucz podstawowy:} \texttt{order\_id}.
    \item \textbf{Klucz obcy:} \texttt{client\_id}.
\end{itemize}

\subsection{Tabela Klienci}
\begin{itemize}
    \item \textbf{Klucz podstawowy:} \texttt{client\_id}.
\end{itemize}

\subsection{Tabela Produkty}
\begin{itemize}
    \item \textbf{Klucz podstawowy:} \texttt{product\_id}.
\end{itemize}

\subsection{Tabela Płatności}
\begin{itemize}
    \item \textbf{Klucz podstawowy:} \texttt{payment\_id}.
    \item \textbf{Klucz obcy:} \texttt{order\_id}.
\end{itemize}

\subsection{Tabela Raporty}
\begin{itemize}
    \item \textbf{Klucz podstawowy:} \texttt{report\_id}.
\end{itemize}
\subsection{Tabela Szczegóły Zamówień}
\begin{itemize}
    \item \textbf{Klucz podstawowy:} \texttt{order\_detail\_id}.
    \item \textbf{Klucz obcy:} \texttt{order\_id}.
    \item \textbf{Klucz obcy:} \texttt{product\_id}.
\end{itemize}


\end{document}
