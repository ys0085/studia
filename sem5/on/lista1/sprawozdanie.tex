\documentclass[12pt,a4paper]{article}
\usepackage[polish]{babel}
\usepackage[T1]{fontenc}
\usepackage[utf8]{inputenc}
\usepackage{graphicx}
\usepackage{amsmath}
\usepackage{amssymb}
\usepackage{float}
\usepackage{hyperref}
\usepackage{geometry}
\geometry{margin=2.5cm}

\title{%
Obliczenia naukowe – Lista nr 1\\[0.5em]
\large Sprawozdanie z laboratorium w języku Julia
}
\author{Krzysztof Kleszcz}
\date{\today}

\begin{document}
\maketitle
\tableofcontents
\newpage

\section*{Wprowadzenie}
Celem laboratorium było zapoznanie się z arytmetyką zmiennopozycyjną zgodną ze standardem IEEE~754 oraz z podstawowymi problemami numerycznymi pojawiającymi się w obliczeniach komputerowych. Wszystkie eksperymenty zostały wykonane w języku \textbf{Julia}, który umożliwia łatwe operowanie typami \texttt{Float16}, \texttt{Float32} oraz \texttt{Float64}.



\section{Zadanie 1 - Rozpoznanie arytmetyki}
\subsection*{Opis problemu}
Celem było wyznaczenie maszynowych wartości \texttt{eps}, \texttt{min} oraz \texttt{max} dla różnych precyzji zmiennopozycyjnych oraz porównanie z wartościami zwracanymi przez funkcje systemowe.

\subsection*{Rozwiązanie}
Zaimplementowano iteracyjne algorytmy:
\begin{itemize}
    \item Dla \texttt{macheps}: start od 1.0, dzielenie przez 2 do momentu, aż $1.0 + \epsilon = 1.0$.
    \item Dla \texttt{min}: analogicznie, od 1.0 w dół aż do $0.0$.
    \item Dla \texttt{max}: zwiększanie wartości, aż \texttt{isinf(x)} zwróci \texttt{true}.
\end{itemize}

\subsection*{Wyniki}
\begin{center}
    \includegraphics[scale=0.75]{results_img/zad1.png}
\end{center}

\subsection*{Interpretacja i wnioski}
Otrzymane wartości są zgodne z normą IEEE~754 i wynikami funkcji \texttt{eps()}, \texttt{nextfloat(0.0)} oraz \texttt{floatmax()}. 
Wartość \texttt{macheps} odpowiada precyzji $\varepsilon$ arytmetyki, a \texttt{min} jest najmniejszą dodatnią liczbą reprezentowalną (związaną z \texttt{MINsub}).


\section{Zadanie 2 - Eksperyment Kahna}
\subsection*{Opis problemu}
Sprawdzenie, czy wyrażenie $3(4/3 - 1) - 1$ daje w przybliżeniu wartość \texttt{macheps}.

\subsection*{Wyniki}
\begin{center}
\includegraphics{results_img/zad2.png}
\end{center}

\subsection*{Wnioski}
Moduł otrzymanego wyniku zgadza się dokładnie z epsilonem maszynowym. 


\section{Zadanie 3 - Rozmieszczenie liczb zmiennopozycyjnych}
\subsection*{Opis problemu}
Sprawdzenie faktu, że liczby w przedziale $[1, 2]$ są rozmieszczone równomiernie z krokiem $\delta = 2^{-52}$ dla typu \texttt{Float64}.

\subsection*{Rozwiązanie}
Sprawdzono wartości \texttt{nextfloat(x) - x} dla najmniejszej liczby w przedziale oraz dla losowej liczby z przedziału.

\subsection*{Wyniki}
\begin{center}
\includegraphics[scale=0.75]{results_img/zad3.png}
\end{center}

\subsection*{Wnioski}
Rozmieszczenie liczb jest proporcjonalne do potęgi dwójki.

\section{Zadanie 4 - Iloczyn odwrotnościowy}
\subsection*{Opis problemu}
Poszukiwano najmniejszej liczby $x$ w przedziale $(1, 2)$ takiej, że $x \cdot (1/x) \neq 1$.

\subsection*{Wyniki}
\begin{center}
\includegraphics{results_img/zad4.png}
\end{center}
Najmniejszą liczbą spełniającą warunek okazało się:
\[
x = 1.0000004876006323 \text{ (Float32)} 
\]
\[
x = 1.0000000022204463 \text{ (Float64)}
\]

\subsection*{Wnioski}
Drobne odchylenie od jedności wynika z utraty precyzji podczas operacji odwrotności i mnożenia.


\section{Zadanie 5 - Iloczyn skalarny}
\subsection*{Opis problemu}
Obliczono iloczyn skalarny dwóch wektorów na cztery sposoby i porównano z wartością dokładną $-1.00657107 \times 10^{-11}$.

\subsection*{Wyniki}
Jako \texttt{delta} oznaczono błąd względny.
\begin{center}
\includegraphics[scale=0.75]{results_img/zad5.png}
\end{center}

\subsection*{Wnioski}
Kolejność sumowania ma istotny wpływ na błędy i w pojedynczej, i w podwójnej precyzji, choć w podwójnej przcyzji dla metod 1 i 2 wynik jest wyraźnie bliższy oczekiwanemu.


\section{Zadanie 6 - Porównanie funkcji $f(x)$ i $g(x)$}
\subsection*{Opis problemu}
Porównano wartości:
\[
f(x) = \sqrt{x^2+1} - 1, \quad g(x) = \frac{x^2}{\sqrt{x^2+1} + 1}
\]
dla $x = 8^{-1}, 8^{-2}, 8^{-3}, \ldots$

\subsection*{Wyniki}
\begin{center}
\includegraphics{results_img/zad6_1.png}
\end{center}
\begin{center}
\includegraphics{results_img/zad6_2.png}
\end{center}

\subsection*{Wnioski}
Dla małych wartości $x$ wyrażenie $f(x)$ traci precyzję z powodu odejmowania podobnych wartości, co prowadzi do dużych błędów względnych.
Wyrażenie $g(x)$ jest numerycznie stabilne - eliminuje utratę cyfr znaczących w odejmowaniu podobnych wartości.


\section{Zadanie 7 - Przybliżenie pochodnej}
\subsection*{Opis problemu}
Dla funkcji $f(x) = \sin x + \cos 3x$ obliczono przybliżoną pochodną w punkcie $x_0=1$ dla $h=2^{-n}$.

\subsection*{Wyniki}
\begin{center}
\includegraphics{results_img/zad7.png}
\end{center}

\subsection*{Wnioski}
Zmniejszanie $h$ początkowo poprawia dokładność przybliżenia, ale po pewnym punkcie, w tym przypadku w $h = 2^{-29}$ błąd zaczyna rosnąć z powodu błędu równowagi wynikającego z ograniczonej precyzji arytmetyki zmiennopozycyjnej. 


\end{document}
