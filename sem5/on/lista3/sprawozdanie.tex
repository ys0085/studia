\documentclass[12pt]{article}
\usepackage{amsmath, amssymb}
\usepackage[T1]{fontenc}
\usepackage{graphicx}
\usepackage{float}
\usepackage{geometry}
\geometry{margin=2.5cm}

\title{Obliczenia Naukowe \\ Sprawozdanie - Lista 3}
\author{Imię i nazwisko}
\date{\today}

\begin{document}

\maketitle

\section*{Wstęp}
Celem sprawozdania jest zaimpmlementowanie trzech metod numerycznych rozwiązywania równań oraz 
rozwiązanie zadań 4-6 dotyczących metod numerycznych:
bisekcji, Newtona oraz siecznych. Dla każdej z metod przedstawiono opis,
implementację, wyniki obliczeń oraz wnioski.

\section{Zadanie 4}
Wyznaczyć pierwiastek równania
\[
\sin x - \left( \frac{1}{2}x \right)^2 = 0
\]
stosując kolejno metody:
\begin{itemize}
    \item bisekcji na przedziale $[1.5,2]$,
    \item Newtona z przybliżeniem początkowym $x_0 = 1.5$,
    \item siecznych z przybliżeniami $x_0=1$, $x_1=2$.
\end{itemize}
Dla każdej metody wykonać obliczenia dla dwóch dokładności:
$\delta = 10^{-5}$ oraz $\varepsilon = 10^{-5}$.

\subsection{Opis problemu}
Równanie nieliniowe:
\[
f(x)=\sin x - \left(\frac{x}{2}\right)^2
\]
ma dokładnie jeden pierwiastek w podanym przedziale. Metody iteracyjne mają na celu przybliżenie jego wartości.

\subsection{Rozwiązanie}
Zastosowano numeryczne implementacje trzech metod. 

\subsection{Wyniki}
\begin{itemize}
    \item Metoda bisekcji: $x \approx 1.9337539672851562$, 16 iteracji
    \item Metoda Newtona: $x \approx 1.933753779789742$, 5 iteracji
    \item Metoda siecznych: $x \approx 1.933753644474301$, 4 iteracje
    \item Prawdziwy wynik: $x = 1.9337537628270212533$
\end{itemize}

\subsection{Interpretacja wyników}
Metoda Newtona i metoda siecznych wykazuje najszybszą zbieżność.
Bisekcja jest najwolniejsza.
Wszystkie metody wykazały bardzo mały błąd.

\subsection{Wnioski}
Wszystkie metody poprawnie odnalazły pierwiastek, jednak metoda bisekcji okazała się najwolniejsza.



% ======================================================================
\section{Zadanie 5}
Znaleźć wartość $x$, dla której przecinają się wykresy
\[
y=3x, \qquad y=e^x.
\]
Metoda bisekcji, dokładność: $\delta=10^{-4}$, $\varepsilon=10^{-4}$.

\subsection{Opis problemu}
Szukamy rozwiązania równania nieliniowego:
\[
f(x) = e^x - 3x.
\]

\subsection{Rozwiązanie}
Wybrano dwa przedziały początkowe $[0,1]$ oraz $[1,2]$ ponieważ:
\[
f(0)=1>0,\quad f(1)=e-3<0, \quad f(2)=e^2-6>0
\]

\subsection{Wyniki}
Rozwiązania przybliżone:
\[
x_1 \approx 0.619140625
\quad
x_2 \approx 1.5120849609375
\]


\subsection{Wnioski}
Metoda bisekcji pozwoliła na wyznaczenie miejsca przecięcia z dowolną dokładnością,
lecz problemem jest wyznaczenie początkowych przedziałów: nie mamy pewności, że to wszystkie rozwiązania.

% ======================================================================
    \section{Zadanie 6}
Znaleźć miejsca zerowe funkcji
\[
f_1(x)=e^{1-x}-x,\qquad
f_2(x)=xe^{-x}
\]
stosując metody bisekcji, Newtona i siecznych.
Wymagana dokładność: $\delta=10^{-5}$, $\varepsilon=10^{-5}$.

Sprawdzić także:
\begin{itemize}
    \item co się stanie przy wybraniu $x_0 \in (1,\infty)$ w metodzie Newtona dla $f_1$,
    \item oraz czy można wybrać $x_0=1$ dla $f_2$.
\end{itemize}

\subsection{Opis problemu}
Szukamy miejsc zerowych dwóch funkcji nieliniowych, stosując trzy metody
i porównując ich zbieżność oraz wrażliwość na punkt startowy.

\subsection{Rozwiązanie}
Dla każdej metody wykonano obliczenia numeryczne. Sprawdzono również,
czy metoda Newtona jest zbieżna przy wskazanych punktach startowych.

\subsection{Wyniki}

Dla $f_1$:
\begin{itemize}
    \item Bisekcja: $x\approx 1.0000038146972656$, przedział $[-5,5]$
    \item Newton: $x\approx 0.9999984358892101$, $x_0 = 0$
    \item Sieczne: $x\approx 1.000002147435368$, $x_1 = -2, x_2 = 2$
\end{itemize}
Dla $f_2$:
\begin{itemize}
    \item Bisekcja: $x\approx -1.9073486328125e-6$, przedział $[-11,10]$
    \item Newton: $x\approx 14.398662765680003$, $x_0 = 2$
    \item Sieczne: $x\approx 14.289550062128848$, $x_1 = -2, x_2 = 2$
\end{itemize}
Metoda Newtona dla $f_1$, $x>1$:
\begin{itemize}
    \item $x_0 = 5$: $x\approx 0.9999996427095682$
    \item $x_0 = 10$: error 1: przekroczono max. iteracje
    \item $x_0 = 20$: error 1: przekroczono max. iteracje
    \item $x_0 >= 100$: error 2: pochodna bliska zeru
\end{itemize}

Metoda Newtona dla $f_2$, $x\ge1$:
\begin{itemize}
    \item $x_0 = 1$: error 2: pochodna bliska zeru
    \item $x_0 = 5$: $x \approx 15.194283983439147$
    \item $x_0 = 10$: $x \approx 14.380524159896261$
    \item $x_0 = 100$: $x = 100$
    \item $x_0 = 1000$: $x = 1000$
\end{itemize}

\subsection{Interpretacja wyników}
\begin{itemize}
    \item Dla $f_1$ metoda Newtona jest zbieżna dla małych $x_0$, lecz dla większych $x_0$ pochodna jest zbyt mała.
    \item Dla $f_2$ punkt $x_0=1$ jest niepoprawny, bo $f_2'(1)=0$. Dla dużych $x_0\ge100$ $f_2(x_0)$ jest równe zeru w arytmetyce Float64, więc algorytm zwraca wartość początkową.
\end{itemize}

\subsection{Wnioski}
Dla funkcji takich jak $f_1$ i $f_2$, które zbiegają do zera w nieskończoności, 
należy poprawnie dobrać dane początkowe.

\end{document}
